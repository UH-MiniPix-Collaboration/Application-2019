\section{Procedures}
\label{sec:Procedures}

\subsection{Decontamination}
\label{subsec:Decontamination}

\subsubsection{Objectives}
Sanitization procedures are critical. They need to be checked and verified to ensure that our samples will not become contaminated. If the samples were to become contaminated it would make any possible bacterial collection data inconsequential.

\subsubsection{Sterilization Preflight}
The payload will be built within the confines of a class 100 clean hood that is located inside of a class 10,000 clean room. Any tools that are used to construct the sampling box will be heat sterilized at \SI{120}{\celsius} for \SI{20}{\minute}. This will be followed by exposing each side of the container to germicidal UV-C (\SI{254}{\nano\meter}) light for \SI{20}{\minute} and then soaked overnight in 91\% isopropyl alcohol to denature proteins in any possible sources of contaminating bacteria. This sterilization method destroys close to 100\% of all organisms and their endospores. To sterilize parts that would otherwise be damaged by the autoclave method they will be cleaned by hand with 91\% isopropyl alcohol to kill microorganisms by denaturing proteins and dissolving the lipid membrane. Following this, the materials will be rinsed with a 95\% ethanol (v/v) solution as an extra precautionary step to ensure complete decontamination. After all parts have dried, the sampling container will be constructed and placed in a gas-porous sterilization pouch and exposed to ethylene oxide (EO) at a concentration of 0.45-0.65 \SIrange{0.45}{0.65}{\milli\gram\per\meter\cubed} at \SI{55}{\celsius} and 30-50 \% RH for \num{4} hours to annihilate any spores and to provide another form of anti-bacterial treatment. The SMITH payload for HASP 2011 was processed in a similar fashion. Once the final HASP integration is ready for sampling and control containers are produced, the chambers will be sterilized and sealed. After the containers are integrated into the rest of the payload, the entire device will be placed in an autoclave bag for transportation.

\subsubsection{Sterilization Post flight}
Before payload descent, we will shut off all of our systesms. By powering down all the systems, the solenoids will seal the sampling container. Each team member involved in the recovery process will wear new latex gloves; cleaned with 91 \% isopropyl alcohol. The payload will remain sealed until decontamination procedures are complete and the sampling containers are ready for processing. The payload will be disassembled under class 100 conditions and all tools used during this procedure will be either heat or 91 \% isopropyl alcohol sterilized. Once in the clean room, the same procedures that were performed preflight will be performed post flight. The sampling box will then be packaged in a heat sterilized plastic outer container and transported back to the University of Houston for analysis.


\subsection{Testing and Integration}

\subsubsection{Vacuum Chamber Testing}
Each subsystem will be tested in-house with a vacuum chamber.
This testing phase will be used to ensure that each component of each subsystem performs as it should while in near-vacuum conditions, so if any problem arises it can be understood and fixed.
The ISS module structure will be tested individually with a pressure sensor to ensure that the module remains pressurized for a full \num{24} hour period which will ensure that the welded joints and the hermetic seal are sufficient.
The high-performance devices (i.e. the MiniPIXs and the RP3) will be thermally tested by performing and collecting data for a full \num{24} hour period.
Next, the full electronics system will be tested to ensure the astrobiology components correctly respond to the changing pressure.
To conclude vacuum testing, all subsytems will be fully integrated into one assembly which will then be tested in near-vacuum conditions for a \num{24} hour period.
The complete system will be considered optimal when both MiniPIX devices are ensured not to overheat, the RP3 is shown not to overheat, the astrobiology system deploys at the proper air pressure, and the astrobiology system seals at the proper air pressure all while remaining below the amperage limit of \SI{2.5}{\ampere}.

\subsubsection{HASP Integration}
During HASP integration, the payload will be tested for proper connection to the HASP systems and be thermally testing at least once more.
The integration team will verify that the EDAC connection distributes power as necessary as well as sends the correct discrete commands at will.
The astrobiology system will be tested without the membrane filters so as to conserve resources.
Any necessary tweaks or changes will be made to properly integrate our payload with the HASP systems.
The system will be considered integrated once the payload appropriately receives power and commands and completes thermal testing.

\subsubsection{Post-Integration Operations}
Once integration is complete, the subsystems will undergo one final check, and the astrobiology system will be sterilized and fitted with the membrane filters. If funds can support it, we will send a small team to New Mexico to oversee the flight operations.

\subsection{Flight Operations}
Each system is autonomous, however, a mission control team will monitor the downlinked data to determine the system status. If the astrobiology system fails to automatically respond or the electronics systems needs a reboot, the corresponding discrete command will be sent.

\subsection{Post-Flight Operations}
If a team is onsite in New Mexico, the data storage systems will be collected and safely stored. Otherwise, the payload will be shipped to our facilities.