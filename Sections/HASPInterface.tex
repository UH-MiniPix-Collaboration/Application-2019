\section{HASP Interface}
\label{sec:HaspInterface}

\subsection{Interfacing with HASP: Serial Uplink and Downlink}
During the duration of our flight the serial uplink commands will be used to configure the collection parameters of the MiniPIX radiation detector. When a command is received from the mission control team, it will be transmitted straight through the DB9 connection into the RPI3 through an RS232 to ttl converter. Then the monitoring script on the RPI3 will verify its content and perform the appropriate action defined in Table~\ref{tab:All-Commands}. The baud rate for serial communication will be set to 4800 bits per second ($bits/sec$). 


\begin{table}[!ht]
  \centering
  \caption{Table of All Uplink Command} 
  \label{tab:All-Commands}
  \bigskip
  \begin{tabular}{c|c|c}
    \multicolumn{1}{c|}{\bfseries Command} & \multicolumn{1}{c|}{\bfseries HEX Uplink Command} & \multicolumn{1}{c}{\bfseries HEX Uplink Argument}\\
    \hline
    \hline
    Begin Acquisitions & 0x01 & N/A \\ 
    Stop Acquisitions & 0x02 & N/A \\ 
    Set Shutter Time & 0x03 & Shutter Time in seconds (1-255) \\ 
    Change Acquisition Mode & 0x04 & HEX Mode Identifier \\
    Device Reset & 0x05 & N/A \\
  \end{tabular}
  \medskip
\end{table}

\begin{table}[!ht]
  \centering
  \caption{Table of Acquisition Modes} 
  \label{tab:Acq-Commands}
  \bigskip
  \begin{tabular}{c|c|c}
    \multicolumn{1}{c|}{\bfseries Acquisition Mode} & \multicolumn{1}{c|}{\bfseries Hex Mode Identifier} & \multicolumn{1}{c}{\bfseries Description} \\
    \hline
    \hline
    Automatic Shutter Time & 0x01 & Shutter time determined automatically based on particle flux \\
    Fixed Shutter Time & 0x02 & Shutter time fixed to set shutter time \\ 
    Counting Mode & 0x03 & Configure MiniPIX to measure counts on the detector only \\
    ToT Mode & 0x04 & Configure MiniPIX to measure time over threshold values \\
    ToA Mode & 0x05 & Configure MiniPIX to measure time of arrival \\
  \end{tabular}
  \medskip
\end{table}


\subsection{Interfacing with HASP: EDAC and DB9 Connections}

The power supply unit (PSU) will use a +30V and ground line from the EDAC connection.  Each pump will use a pair of +30V and ground lines respectfully.  The last available pair of +30V and ground line will be used to power the solenoids.  The pumps and solenoids will be wired to a relay which is controlled from the Arduino. 

We will use discrete commands in this mission.  Two of the commands will be used to turn the astrobioly system on and off.  This is only in case of emergency if we notice that the system is not responding to our serial commands.  The other two discrete channels will be used to turn on and off the power of RESU in case that we notice an issue.

\begin{table}[!ht]
  \centering
  \caption{Table of all discrete commands to be used during flight} 
  \label{tab:Dis-Commands}
  \bigskip
  \begin{tabular}{c|c|c|c}
    \multicolumn{1}{c|}{\bfseries Command} & \multicolumn{1}{c|}{\bfseries Purpose} &  \multicolumn{1}{c|}{\bfseries EDAC Pin} & \multicolumn{1}{c}{\bfseries Description} \\
      \hline
      \hline
      Discrete 1 & Astro. System ON & f & Initiates the pumps and collection systems \\
      Discrete 2 & Astro. System OFF & n & Shuts down the pumps and causes collection systems to retract \\
      Discrete 3 & RESU ON & h & Powers up RESU \\
      Discrete 4 & RESU OFF & p & Shuts down RESU \\
  \end{tabular}
  \medskip
\end{table}

\subsection{Interfacing with HASP: Serial Downkink}
For the duration of the flight the serial downlink will be used to downlink the temperature and other various radiation statistics that will be computed directly on the RP\num{3} as data is collected. It will also be used to downlink messages regarding the status of the payload, the command uplink status and error messages. The data packets will be human readable so they can be analyzed directly as they are received from HASP. The packets will be delineated by new lines and we will use a simple one character header to differentiate between data packets and message packets. The format of the packets is still potentially subject to change but the preliminary design is outlined below.



\begin{table}[!ht]
\centering
\caption{Data Packet Format} 
\label{tab:DataPacket}
\bigskip
\begin{tabular}{|c|c|}
\hline
\multicolumn{1}{|c|}{\bfseries Data} & \multicolumn{1}{c|}{\bfseries Size (bytes)}\\
\hline
    d						& 2		\\ \hline 
    MiniPIX Temperature 			& 6		\\ \hline
    Dose Rate					& 6		\\ \hline %two pumps now
    Average Counts   				& 5		\\ \hline
    Newline					& 1		\\ \hline
    Total Packet Size				& 20 bytes		\\ \hline
    
\end{tabular}
\medskip
\end{table}

\begin{table}[!ht]
\centering
\caption{Message Packet Format} 
\label{tab:DataPacket}
\bigskip
\begin{tabular}{|c|c|}
\hline
\multicolumn{1}{|c|}{\bfseries Data} & \multicolumn{1}{c|}{\bfseries Size (bytes)}\\
\hline
    m						& 2		\\ \hline 
    Message Size 				& 6		\\ \hline
    Message					& String of length Message Size		\\ \hline %two pumps now
    Newline					& 1		\\ \hline
    Total Packet Size				& Variable		\\ \hline
    
\end{tabular}
\medskip
\end{table}


