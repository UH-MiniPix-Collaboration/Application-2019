\subsection{Radiation}
\label{subsec:RadiationBackground}


\subsubsection{Cosmic Rays}
Understanding the biological effect of radiation on humans is cruicial for successful space flight.
By observing the environment in which astronauts reside, information regarding the conditions which induce the biological changes can be gathered.
Among this information is dosage of various particles.
In the upper atmosphere, primary particles, which consist largely of protons (p\textsuperscript{+}) and ionized atomic nuclei \cite{Frank}, collide with the nuclei of air molecules and can induce air showers.
These air showers produce secondary particles, which consist of positrons/electrons (e\textsuperscript{$\pm$}), neutrons(n), muons($\mu$\textsuperscript{$\pm$}), pions ($\pi$\textsuperscript{$\pm$}), and various other hadrons.
The primary particles of interest are Galactic Cosmic Rays (GCRs), which originate outside of the solar system but within the Milky Way Galaxy and typically have been accelerated to nearly the speed of light \cite{GCRs}.

The results from the previous missions have confirmed the successful application of the MiniPIX as a dosimeter through the idenification of the Regener-Pfotzer Maximum. This maximum is the altitude at which ionizing radiation reaches its peak \cite{Regener}. The value of this altitude depends on various environmental factors, but it is around \SI{18}{\kilo\meter}. With this confirmation, the MiniPIX will remain the primary tool we will use to measure radiation.

Primary particles can induce an air shower by colliding with the nucleus of any atom or particle.
This poses a potential threat to astronauts residing in spacecrafts, as the materials in the spacecraft's structure can induce an air shower and thus bombard astronauts with radiation.
By using a MiniPIX to measure the radiation outside of as well as within a structure emulating a module from the Internation Space Station, we can observe the environment that current astronauts live in and how the structure changes the radiation field.
Additionaly, the measurements outside of the structure will give information as to the conditions in which atmospheric extremophiles reside.

\subsubsection{Organic Solar Cells}

	\subsection{Introduction}
	Photovoltaic devices has been a staple in space energy harvesting since the launch of Vangaurd 1 in 1958. Since then, silicon, gallium arsenide, indium phosphide, cadmium telluride, and other III-IV systems have played the key roles. When choosing a system to gather solar energy, the only economic factors are the initial cost of the arrays, and the cost of transporting the system to orbit. To reduce this cost, we want solar cells with the highest specific power, which is defined as the W/kg. Recent studies have shown organic solar cells (OSC) with a specific powers of 10 W/g, which is why OSCs are an active area of research for space energy generation. Organic solar cells are thin film, light weight, flexible, and can be produced using roll to roll printing, making it ideal for space applications. In this experiment, we are exploring the effects of stratospheric conditions on organic solar cells, with a particular focus on the effects of cosmic ray bombardment in conjunction with the MiniPIX system, as well as to investigate the radiation hardness of different types of organic solar cells and demonstrate their reliability upon entry and exit of the stratosphere. In the following sections we will explain the basic workings of an OSC and explain some of the benifits and hinderences of a space like environment on the OSC, then explain the methodology of our experiment.
	\cite{Space organics}
	\subsection{Working principals}

	
	OSCs use earth abundant, carbon based semiconductors, such as conjugated polymers(as donors) and fullerene derivatives(as acceptors). Where as in mono-crystalline solar cells the valance and conduction bands are used to determine the energy band gap, OSCs model the difference between the highest occupied molecular orbit(HOMO) and the lowest unoccupied molecular orbit(LUMO) as equivalent to this energy band gap. When donor(D) and acceptor(A) layers meet, their Fermi levels match up and band bending occurs, which creates a small electric field, or built in potential (Vbi). Charge generation occurs when excitons, an electron/hole pair bound together by the Coulomb force, are generated in the donor material. This exciton will travel towards to D-A interface, where the aforementioned Vbi will seperate the exciton into free charges given sufficient conditions.  Exciton transportation inside of the bulk is hindered to 5-50nm before recombination occurs depending on material and structure, while the average thickness of an OSC is 50-100nm, thus the architecture of the active layer is of much importance. Instead of using a bi-layer of n-type and p-type semiconductors, OSCs use the bulk hetero-junction (BHJ), which is a blend with A-D regions that are within the estimated diffusion length.  
	
A newer type of organic-inorganic blend solar cell which has gained a lot  attention is the perovskite solar cell (PSC). This solar cell has a novel active layer comprised of organometallic molecules which exhibit the perovskite crystal structure ABX3. The most commonly used perovskite material is methylamoniumm lead trihalide (CH3NH3PbX3, MAPX), where lead can be swapped for tin, and halides such as boron, iodine, and chlorine are employed. Fabrication of the perovskite solar cell is highly attractive as it can be accomplished with wet chemistry, roll to roll printing, or vapor deposition. Some key features of the perovskite structure is that the bandgap of the material is tunable via the halide content and that the diffusion length of holes and electrons is ~1 micron, making them perfect for roll to roll thin film printing. The Shockley–Queisser limit is about 31 percent under an AM1.5G solar spectrum at 1000W/m2, for a Perovskite bandgap of 1.55 eV, which is near the theoretical maximum of 33.7 percent, and reports of perovskite cells with PCE as high as 22 percent have already been fabricated. As a newer material there is still much unknown about the physics of the perovskite solar cell, such as if charges occur as bound excitons or free charges.

	Space transport involves high doses of radiation, large fluctuations of temperatures, and extreme vacuum conditions. The need to study the effects of a space like environment on organic electronic devices is paramount as advances in the organic semiconductor field continue and space exploration becomes more standard. In this study, we propose to mount four different types of solar cells onto our payload and monitor their performance. We will be using prefabricated small molecule based OSCs, self fabricated polymer based OSCs, self fabricated PSCs, and prefabricated amorphous silicon solar cells. Each of these is a lightweight, flexible, and thin solar cell ideal for space use. Additionaly, effect to the crystal structure of the perovskite based cell is of particular interest, as cosmic ray bombardments and fluxtuations of temperature can be determental. 
	


%\subsubsection{Solar Radiation}
%The ultraviolet (UV) spectrum is composed of UVC (\SIrange{200}{280}{\nano\meter}) with only \SI{0.5}{\percent} of the entire solar spectrum, 1.5\% of
%UVB (\SIrange{280}{315}{\nano\meter}) and UVA (\SIrange{315}{400}{\nano\meter}), which contributes to \SI{6.3}{\percent}~\cite{uv_irradiance}. UVB and UVC are the main contributors in highly lethal solar radiation to microorganisms\cite{UVonDNA}. DNA is prone to high absorption levels at those wavelengths, often causing inactivation and mutation.  Therefore, understanding the exposure of microbes in to UV radiation is quite important.

%In the atmosphere ionizing radiation does not always increase with altitude. As
%primary GCRs such as high energy protons, alpha particles, and 
%heavy ions collide with atoms in the atmosphere they begin to shatter into secondary
%particles such as neutrons, pions, electrons and muons which causes a peak in 
%ionizing radiation at around \SI{18}{\kilo\meter}. This peak is known 
%as the Regener-Pfotzer Maximum\cite{regener} and shows that with increasing atmospheric density
% ionizing radiation increases until peaking high in the stratosphere and then decreases rapidly as you 
%reach the surface of the earth.



%Expand upon Regener-Pfotzer Maximum. Refer to other papers to validate the range. 
%The intensity of GCR peaks within a range of about
%\SI{18}{\kilo\meter} to about \SI{22}{\kilo\meter} []. This range,where the production of ionizing radiation reaches its
%peak is known as the Regener-Pfotzer Maximum\cite{regener} . The Regener-Pfotzer Maxiumum is unique and is dependent on
%location and time of the year, as it is determined by a number of
%factors, which include but are not limited to the strength of earth's
%electromagnetic field, atmospheric composition (specifically ozone
%content(?){\bf The intensity of the cosmic ray flux and the secondary
%  environment vary inversely with the solar cycle due to the
%  interaction of the earths electromagnetic field. In addition, the
%  sporadic solar events that occur in short busts can increase the
%  primary particle flux periodically (hours to days) can in fact
%  enhance the atmospheric radiation several orders of magnitude in
%  scale.}), the sun's relative position, and maximum solar elevation
%[]. The combination of these affects results in a variability in the
%location of the maximum as well as the existene of this maximum as
%ooposed to an ever-increasing intensity.

