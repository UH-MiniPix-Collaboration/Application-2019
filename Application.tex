\documentclass[aps,superscriptaddress,floatfix,nofootinbib,showpacs,amsmath,amssymb,altaffilletter,floatfix,onecolumn]{revtex4-1}

\input{def.tex}
\usepackage{fancyhdr}
\usepackage{array}
\usepackage{enumitem}
\usepackage{graphicx}
\usepackage{wrapfig}
\usepackage{float}
\usepackage[title]{appendix}

\renewcommand{\headrulewidth}{0pt}
\renewcommand{\thepage}{}
\renewcommand{\thepage}{\arabic{page}}
\renewcommand\thesection{\arabic{section}}
\renewcommand\thesubsection{\thesection.\arabic{subsection}}
\renewcommand\thesubsubsection{\thesubsection.\arabic{subsubsection}}
\newcommand{\hasplogo}{\includegraphics[width=0.08\linewidth]{Figures/logo.pdf}} %Header
\setlength\extrarowheight{4pt}

\makeatletter
\def\p@subsection{}
\makeatother
\makeatletter
\def\p@subsubsection{}
\makeatother

\fancyhf{}
\pagestyle{fancy}
\headheight 50pt
\lhead{\hasplogo} %Header
\chead{\vspace*{-2cm}\Large\textbf{HASP Payload Application 2019}} %Header
\rfoot{\thepage}

\parskip = 6pt %changes spacing between paragraphs


\begin{document}

\title{University of Houston HASP 2019 Application}

\begin{abstract}
  \begin{center}
    {\bf Abstract}
  \end{center}

  By learning from the methodology imposed by the previous SORA missions, the SORA 3 payload will feature improved systems to sample for extremophilic microorganisms and observe the stratospheric radiation environment.
  An overhauled system to capture microorganisms is being implemented by learning from what did and did not work on the previous SORA missions.
  Furthermore, the radiation system is expanding to include a container that will closely mimic an ISS module in terms of material structure in hopes of observing the radiation environment to which astronauts are exposed.
  In addition, the SORA 3 payload will implement a new system featuring organic photovoltaic cells in the interest of observing the change in performance and molecular structure of the cells as a result of prolonged exposure to intense radiation and a near space environment.
  The first two missions have established a technological foundation from which SORA 3 can expand through exploring more deeply the topics and challenges of the first missions. SORA 3 will continue to use hobby electronics to test the bounds of commerically available technology.
  
  \newpage %Breaks page for the Table of Contents.
\end{abstract}

\newcommand{\Physics}{College of Natural Sciences and Mathematics, Department of Physics, University of Houston, Houston, TX 77204, USA}
\newcommand{\CS}{College of Natural Sciences and Mathematics, Department of Computer Science, University of Houston, Houston, TX 77204, USA}
\newcommand{\Biology}{College of Natural Sciences and Mathematics, Department of Biology, University of Houston, Houston TX, 77204, USA}
\newcommand{\BME}{Cullen College of Engineering, Department of Biomedical Engineering, University of Houston, Houston TX, 77204, USA}
\newcommand{\Chemical}{Cullen College of Engineering, Department of Chemical and Biomolecular Engineering, University of Houston, Houston TX, 77204, USA}
\newcommand{\Electrical}{Cullen College of Engineering, Department of Electrical Engineering, University of Houston, Houston TX, 77204, USA}
\newcommand{\Mechanical}{Cullen College of Engineering, Department of Mechanical Engineering, University of Houston, Houston TX, 77204, USA}


%--- Add authors in the order they should appear
\author{R.~Masek}\affiliation{\Physics}
\author{J.~Patel}\affiliation{\Physics}
\author{M.~Butowicz}\affiliation{\CS}
\author{T.~Hill}\affiliation{\Physics}
\author{C. Amay}\affiliation{\CS}
\author{J. Arguello}
\author{A. Boggs}\affiliation{\Physics}
\author{C. Bush}\affiliation{\BME}
\author{A. Elsaadi}
\author{K. Fleming}\affiliation{\Chemical}
\author{S. Gupta}
\author{D. Howard}\affiliation{\Electrical}
\author{E. Humble}\affiliation{\CS}
\author{A. Kalavadwala}\affiliation{\Mechanical}
\author{K. Ngo}\affiliation{\Physics}
\author{C. Rimbau}\affiliation{\Biology}
\author{J. Tristan}\affiliation{\Mechanical}
\author{H. Trong}\affiliation{\Mechanical}
\author{A. Vega}\affiliation{\Mechanical}
\author{S.~George}\affiliation{\Physics}
\author{I.~Wilson}\affiliation{\Biology}
\author{D.~Pattison}\affiliation{\Biology}
\author{P.~Gunaratne}\affiliation{\Biology}
\author{A.~L.~Renshaw}\affiliation{\Physics}



\setlength{\parindent}{1em}
\setdefaultleftmargin{1em}{1em}{}{}{}{}
\setcounter{page}{0}\thispagestyle{empty}
\maketitle
\onecolumngrid
\setcounter{tocdepth}{2}
\setcounter{page}{0}\thispagestyle{empty}
\tableofcontents
\setcounter{page}{0}\thispagestyle{empty}
\newpage

%---
%Section: Mission Overview w/ Subsections Mission Statement and Hypotheses
\section{Mission Statement and Objectives}
\label{sec:Introduction}


The 2019 University of Houston (UH) HASP team has set out to expand upon the previous missions \cite{SORA1}\cite{SORA2} from the UH team by applying the knowledge and technology that has accrued.
The methodology of the first two missions has provided a proof of concept for handling tools, such as the MiniPIX particle detector, in extreme environments.
This will serve as the platform for the 2019 mission.

The main objectives of the 2019 mission is to study the exposure to ionizing radiation that organisms face in the upper atmosphere and sample these organisms and bring them to the surface.
The organisms in question are microbial extremophiles that naturally reside in this region of the atmosphere.
As such, our payload will be a continuation of the work performed by the previous UH HASP teams and will be called the Stratospheric Organism and Radiation Analyzer (SORA) 3.

However, we are also interested in the humans that reside in such harsh environments.
We will construct a thermally-controlled, pressurized mock-up International Space Station (ISS) module containing a scintillated MiniPIX.
The pressure will be held constant at \SI{1}{atm} to replicate conditions aboard the ISS.
This allows us to study the particle cascades induced by the materials of the module, which will give information regarding the dose that astronauts are exposed to while in space.
The payload will also contain a second, unscintillated MiniPIX that will reside outside of the ISS module, recording the outside radiation exposure.
This second MiniPIX will act as a control for the ISS module, but also allows us to study the environment that extremophiles are exposed to.
In addition to the MiniPIX, we will test the performance of organic photovoltaic cells in stratospheric conditions.
Organic photovoltaic cells have promising applications for space-based missions due to their inexpensive, lightweight, and durable nature relative to their non-organic counterparts.
In tandem to the radiation study, we will collect samples of the microbial life through the use of a passive collection system.
This system will consist of a T-shaped arm covered with filters that will only be exposed to the outside while at float altitude.
We will then bring the uncontaminated, collected samples down to the surface for analysis.

\noindent The upcoming mission has the following scientific objectives:

\noindent \textbf {Primary Scientific Objectives:}
\begin{enumerate}
\item Capture microorganisms in the upper atmosphere at altitudes of approximately \SIrange{30}{41}{\kilo\meter} using a method not previously used by the UH HASP team. 
\item Study the cosmic and terrestrial radiation that extremophiles and astronauts are exposed to.
\item Observe the influence of intense radiation on the structure and performance of organic photovoltaic cells.
\end{enumerate}

\noindent \textbf {Secondary Scientific Objectives:}
\begin{enumerate}
\item Testing the newly developed astrobiology hardware in flight and establish a more reliable method for collecting microbes in extreme environments at high-altitude.
\item Implement a redundant storage system for all data recorded.
\item Establish a methodology which allows two or more MiniPIX devices to be used in conjunction.
\item Study and test the performance of organic photovoltaic solar cells in the intense radiation environment.
\end{enumerate}

\noindent \textbf {Engineering Objectives:}
\begin{enumerate}
\item Develop a new astrobiology collection mechanism that is favorable at high altitude.
\item Construct a structure resembling an ISS module as accurately as possible.
\item Use a scintillator to detect thermal neutrons using a MiniPIX.
\item Analyze MiniPIX data in real time and downlink relevant information.
%\item Test an improved enclosure against impacts and harsh environments.
\end{enumerate}


\subsection{Hypothesis and Objectives}
\label{subsec:Hypothesis-Objectives}
\begin{enumerate}
\item By comparing our flight sample to previous missions \cite{SORA1}\cite{SORA2}, the newly developed passive astrobiology system can be quantitively and qualitatively be compated to previous methods.
  \begin{enumerate}
  \item Objective: Sample a comparatively larger volume of air using passive design elements rather than an active pump design at float altitude.
  \item Objective: Compare the effectiveness in rotation of the sampling filters to those that are stationary.
  \end{enumerate}
\item Background sampling and rigorous sterilization procedures will help rule out contamination as a possible source of error.
  \begin{enumerate}
  \item Minimize outside contamination of the entire astrobiology system.
  \item Objective: Retain a sterile environment for the entirety of the balloon flight.
  \item Objective: Sample a minimum volumetric amount of air at target altitude for the duration of the float phase (approximately 15 to 18 hours).
  \item Objective: Take background samples using Fluropore Membrane filters at the various locations where the payload will be.
  \end{enumerate}
\item The radiation environment within the ISS module will be noticeably different than that outside the module in terms of particle type and concentration. 
  \begin{enumerate}
  \item Objective: Characterize the radiation environment within the ISS module by particle types and dose.
  \item Objective: Successfully identify neutron interaction with the scintillated MiniPIX.
  \item Objective: After capturing samples, analyze data and compare biological effects to similar genotypes found on Earth's surface.
  \end{enumerate}
\item The organic photovoltaic cells exposed to stratospheric conditions will under perform cells which have remained on Earth.
  \begin{enumerate}
  \item Objective: Compare quantitative and qualitative properties of the cells such as fill factor, efficiency, and physical structure of post-flight cells to cells that have remained on Earth.
  \item Objective: Using microscopy techniques, analyze the influence of stratospheric radiation on the structure of the cells.
  \end{enumerate}
\end{enumerate}
  
\vspace*{-0.5cm}
\section{Background}
    \input{Astrobiology/IntroAstro.tex}
    \subsection{Radiation}
\label{subsec:RadiationBackground}


\subsubsection{Cosmic Rays}
Understanding the biological effect of radiation on humans is cruicial for successful space flight.
By observing the environment in which astronauts reside, information regarding the conditions which induce the biological changes can be gathered.
Among this information is dosage of various particles.
In the upper atmosphere, primary particles, which consist largely of protons (p\textsuperscript{+}) and ionized atomic nuclei \cite{Frank}, collide with the nuclei of air molecules and can induce air showers.
These air showers produce secondary particles, which consist of positrons/electrons (e\textsuperscript{$\pm$}), neutrons(n), muons($\mu$\textsuperscript{$\pm$}), pions ($\pi$\textsuperscript{$\pm$}), and various other hadrons.
The primary particles of interest are Galactic Cosmic Rays (GCRs), which originate outside of the solar system but within the Milky Way Galaxy and typically have been accelerated to nearly the speed of light \cite{GCRs}.

The results from the previous missions have confirmed the successful application of the MiniPIX as a dosimeter through the idenification of the Regener-Pfotzer Maximum. This maximum is the altitude at which ionizing radiation reaches its peak \cite{Regener}. The value of this altitude depends on various environmental factors, but it is around \SI{18}{\kilo\meter}. With this confirmation, the MiniPIX will remain the primary tool we will use to measure radiation.

Primary particles can induce an air shower by colliding with the nucleus of any atom or particle.
This poses a potential threat to astronauts residing in spacecrafts, as the materials in the spacecraft's structure can induce an air shower and thus bombard astronauts with radiation.
By using a MiniPIX to measure the radiation outside of as well as within a structure emulating a module from the Internation Space Station, we can observe the environment that current astronauts live in and how the structure changes the radiation field.
Additionaly, the measurements outside of the structure will give information as to the conditions in which atmospheric extremophiles reside.


\subsubsection{Organic Solar Cells}
	Harvesting energy in space is a huge concern, but with a constant supply of photons streaming from the sun, solar power is an obvious, and convenient choice for energy generation. When choosing a system to gather solar energy, one of the largest concerns is getting the system out out to space. The largest economic factor here is weight, which is why organic solar cells (OSC) are an active area of research for space energy generation. Being extremely light weight, the initial cost of transportation is drastically less than traditional solar cells, and with a high degree of flexibility, OSCs offer a larger surface area for application. 
	
	OSCs use earth abundant, carbon based semiconductors, such as conjugated polymers(as donors) and fullerene derivatives(as acceptors). Where as in mono-crystalline solar cells the valance and conduction bands are used to determine the energy band gap, OSCs model the difference between the highest occupied molecular orbit(HOMO) and the lowest unoccupied molecular orbit(LUMO) as equivalent to this energy band gap. When donor(D) and acceptor(A) layers meet, their Fermi levels match up and band bending occurs, which creates a small electric field, or built in potential (Vbi). Charge generation occurs when excitons, an electron/hole pair bound together by the Coulomb force, are generated in the donor material. This exciton will travel towards to D-A interface, where the aforementioned Vbi will seperate the exciton into free charges given sufficient conditions.  Exciton transportation inside of the bulk is hindered to 5-50nm before recombination occurs depending on material and structure, while the average thickness of an OSC is 50-100nm, thus the architecture of the active layer is of much importance. Instead of using a bi-layer of n-type and p-type semiconductors, OSCs use the bulk hetero-junction (BHJ), which is a blend with A-D regions that are within the estimated diffusion length.  
	
	

	Space transport involves high doses of radiation, large fluctuations of temperatures, and extreme vacuum conditions. The need to study the effects of a space like environment on organic electronic devices is paramount as advances in the organic semiconductor field continue and space exploration becomes more standard. The effect of cosmic ray bombardment on perovskite cells is of particular interest as the crystal morphology may be affected. 


%\subsubsection{Solar Radiation}
%The ultraviolet (UV) spectrum is composed of UVC (\SIrange{200}{280}{\nano\meter}) with only \SI{0.5}{\percent} of the entire solar spectrum, 1.5\% of
%UVB (\SIrange{280}{315}{\nano\meter}) and UVA (\SIrange{315}{400}{\nano\meter}), which contributes to \SI{6.3}{\percent}~\cite{uv_irradiance}. UVB and UVC are the main contributors in highly lethal solar radiation to microorganisms\cite{UVonDNA}. DNA is prone to high absorption levels at those wavelengths, often causing inactivation and mutation.  Therefore, understanding the exposure of microbes in to UV radiation is quite important.

%In the atmosphere ionizing radiation does not always increase with altitude. As
%primary GCRs such as high energy protons, alpha particles, and 
%heavy ions collide with atoms in the atmosphere they begin to shatter into secondary
%particles such as neutrons, pions, electrons and muons which causes a peak in 
%ionizing radiation at around \SI{18}{\kilo\meter}. This peak is known 
%as the Regener-Pfotzer Maximum\cite{regener} and shows that with increasing atmospheric density
% ionizing radiation increases until peaking high in the stratosphere and then decreases rapidly as you 
%reach the surface of the earth.



%Expand upon Regener-Pfotzer Maximum. Refer to other papers to validate the range. 
%The intensity of GCR peaks within a range of about
%\SI{18}{\kilo\meter} to about \SI{22}{\kilo\meter} []. This range,where the production of ionizing radiation reaches its
%peak is known as the Regener-Pfotzer Maximum\cite{regener} . The Regener-Pfotzer Maxiumum is unique and is dependent on
%location and time of the year, as it is determined by a number of
%factors, which include but are not limited to the strength of earth's
%electromagnetic field, atmospheric composition (specifically ozone
%content(?){\bf The intensity of the cosmic ray flux and the secondary
%  environment vary inversely with the solar cycle due to the
%  interaction of the earths electromagnetic field. In addition, the
%  sporadic solar events that occur in short busts can increase the
%  primary particle flux periodically (hours to days) can in fact
%  enhance the atmospheric radiation several orders of magnitude in
%  scale.}), the sun's relative position, and maximum solar elevation
%[]. The combination of these affects results in a variability in the
%location of the maximum as well as the existene of this maximum as
%ooposed to an ever-increasing intensity.



%Payload Design and Operation
\section{Payload Systems and Operation}
\label{sec:PayloadSystems}
    \label{sec:Hardware}

SORA 3 will implement an overhauled astrobiology system that will collect larger samples of stratospheric organisms as well as require less amperage, allowing us to expand the radiation system and reduce the chance of electrical failure. The radiation system will apply the techniques developed over the previous missions by consisting of a scintillated MiniPIX within a pressurized module modelled after an ISS module. Additionally, various types of organic photovolatic cells will be flown in order to study their performance in harsh stratospheric conditions.

    \subsection{Astrobiology System}
\label{sec:AstrobiologySystem}

\subsubsection{Design and Operation}
The collection assembly will be designed as a one mechanized enclosure as shown below. The rotating arm will be raised out of the collection container and began rotating once float altitude has been reached. The raising of the rotating arms will be done by a L-16 linear servo motor. The rotation of the filter arm will be provided by the rotational motor mounted onto the filter arm. The lid on the top of the linear servo will be the lid. The filters are Fluropore Membrane filters and will be mounted on the ends of the rotating arm. The arms will spin at 80 RPM for the duration of float conditions. When it is time for the payload to descend, a command will be sent to retract the rotating mechanism back into the clean box. 

\begin{figure}[!h] 
	\begin{center}
		\includegraphics[scale=.8]{}
		\caption{{\bf Left:} Cross-section view of the clean box with experimental and control containers. {\bf Right:} KNF N84-4 commercial gas-sampling diaphragm vacuum pump. }
		\label{fig:Pump}
	\end{center}
\end{figure} 
%\subsection{Astrobiology Methods}
%\label{sec:Astrobiology Methods}
\subsubsection{Pre-Flight Preparation}
The rotating arm mechanism will be tested in low pressure and in various mounting positions. Once it is certain that the arm functions in the intended conditions during flight. We will take it to the clean room to be sterilized.  

The clean box that houses the arm  will autoclaved. All tools used in the assembly of the clean box will either be autoclaved or soaked in a \SI{70}{\percent} ethanol solution inside of a clean room. Each person who enters the clean room will be garbed in a lab coat, goggles, hair net and latex gloves after thoroughly washing their hands in a \SI{70}{\percent} ethanol solution.All wires within in the box will be soaked in the \SI{70}{\percent} ethanol solution and the rotating arm mechanics will be powered on and retracted the box. The box will then be integrated into the payload. 
\subsubsection{Post-Flight Procedures}

Once the payload is retrieved, the intact clean box needs to be removed and placed inside of a cooler with ice to be then transported to The University of Houston and placed in cold storage at \SI{4}{\celsius}. All equipment used in the filtration process will be either autoclaved or taken from previously unopened sanitized packaging. The autoclaved, pre-sanitized items and the clean box will then be washed in a \SI{70}{\percent} ethanol solution before they are placed inside a SterilGARD e3 Class II Biological Safety Cabinet (the Cabinet). The cabinet has a laminar flow air barrier and UV lights built into the ceiling for decontaminating the workspace prior to use.The Fluropore Membrane filters will be processed through a DNA extraction kit and the remaining sample fluid will be stored in the cold storage for in-house 16S ribosomal RNA sequencing through the University of Houston sequencing team.  


    \subsection{Radiation Monitoring System}
\label{sec:RadiationDesign}

\subsubsection{MiniPIX Detector}
The MiniPIX detector is a silicon-based hyrbid pixel detector. The device is built by ADVACAM \cite{Advacam}  and utilizes technology developed by the Medipix Collaboration at CERN \cite{Medipix}. The sensor consists of a \num{256}x\num{256} array of pixels and a pitch of \SI{55}{\micro\meter}. A USB 2.0 connection is used to interface with the device, which provides power and data output. The primary use of the detector will be to characterize cosmic radiation by the type of particle and its incident energy.
%The device is capable of operating in the following three different modes: single particle counting, Time-over-Threshold, and Time-of-Arrival.

%Mention the solar cycle and how we'll compare to data from the previous missions

\begin{figure}[H]
  \begin{minipage}[c]{0.40\linewidth}
    \includegraphics[width=\linewidth]{Figures/Minipix2.png}
    \caption{Picture of a MiniPIX particle detector~\cite{Advacam}.}
    \label{fig:Minipix}
  \end{minipage}
  \hfill
  \begin{minipage}[c]{0.45\linewidth}
    \includegraphics[width=\linewidth]{Figures/MinipixLayers.png}
    \caption{Layers that makeup the MiniPIX sensor.}
    \label{fig:MinipixLayers}
  \end{minipage}
\end{figure}

When an ionizing particle hits the sensor, electron-hole pairs accumulate within the semiconducting material. The electronics reads-out the pairs by depleting the silicon with a bias voltage. If a charge is above the set threshold, the charge is counted. The energy deposited in a pixel can be determined from the back-plane pulse amplitude. Particles incident on the sensor appear as pixel clusters, which is defined as a continuous area of activated pixels. By analyzing these clusters, the incident particle can be identified. The morphology of the cluster gives information regarding the type of particle as well as the angle at which the particle was incident upon the detector. 

\begin{figure}[h]
    \includegraphics[scale=1, width=.75\textwidth]{Figures/Tracks.png}
    \caption{Examples of different track types from data collected by a MiniPIX.}
    \label{fig:Tracks}
\end{figure}

\subsubsection{Calibration}   
It is necessary to calibrate the sensor. The calibration procedure is rather sophisticated, so the calibration will be applied by Dr. Stuart P. George at the University of Houston. Dr. George is a member of the Medipix Collaboration. The general outline of the calibration is as follows. A source calibration is applied by using a \SI{60}{\keV}{\ce{^241Am} decary line, \ce{Sn} Flourescence and \ce{^55Fe} gamma rays. The pixel detector consists of \num{65536} silicon p-n dioded, each containing its own individual processing circuit. The response of each pixel can never to identical to one another, so a calibration mut be performed on each pixel. Dr. George will calibrate each pixel energy threshold from DAC counts to energy \cite{StuartThesis}. The threshold of the sensor will be set to \SI{4}{\keV}, which sufficiently filters out background noise. The bias voltage of the sensor will be set to \SI{200}{\volt}, which guarantees that the silicon is completely depleted while reading out the charge.
  
\subsubsection{Collection Parameters}
% Put the settings for the minipix shutter time, bias voltage etc. here
ADVACAM has developed a Python API to configure the device and its acquisition parameters. The shutter speed of the device controls the rate at which data is output from the device. The shutter speed controls the exposure time of the sensor before saving the data to a frame. Once the data is saved, the device enters manufacturer-set dormant period, which lasts for approximately two seconds. This time will be utilized as a cool-down period for the device. The value of the shutter speed is highly dependent on the application of the sensor. Within the context of HASP, we expect high radiation exposure which will require a quicker shutter speed. This is to the fact that if the shutter speed is too long, the sensor become overcrowded and clusters become difficult or impossible to analyze. However, if the shutter speed is too short, there will be many frames with little to no data, which will use a large amount of storage space. Based off of previous SORA missions, a shutter speed of a few seconds is sufficient for our application. The exact shutter speed will be determined through testing and may be different for each device we use.

\subsubsection{Data Format}
The data frames are stored as plain text values where each value corresponds to a value from a pixel on the pixel array. In a separate file, metadata such as acquisition type, shutter speed, bias voltage, and other parameters are recorded for each frame. Both the data file and the metadata file will be stored on the RP\num{3} and the various redundant storage systems we implement. The MiniPIX data from previous missions used about \SIrange{1}{2}{\giga\byte}. For the 2019 mission, we expect about double that value due to our use of two MiniPIX devices.

\subsubsection{Structure}
The MiniPIX outside of the ISS module will be contained within a small case plastic case to protect the device from atmospheric moisture. Thermal paste will be used to fix the MiniPIX on a small block of aluminum, which will behave as a heat sink. This simple setup has proven to work on previous missions and is shown in Figure \ref{fig:CaseAssembly}.
\begin{figure}[h]
    \includegraphics[scale=1, width=.25\textwidth]{Figures/MinipixCaseAssembly.pdf}
    \caption{Examples of different track types from data collected by a MiniPIX.}
    \label{fig:CaseAssembly}
\end{figure}

\subsubsection{ISS Module}
\label{subsec:ISSModule}
As part of the radiation subsystem, the payload will include a capsule intended to mimic the structure of one of NASA's modules on the ISS.
The method deployed by NASA and other agencies is called ``Stuffed Whipple'' shielding and is deployed in ISS modules that are at highest risk of impact \cite{NASAShielding}.
This module will remain at atmospheric pressure throughout the entire flight in attempt to model the environment as closely as possible.
Inside the capsule will be a scintillated MiniPIX that will record the radiation environment within the module.
The scintillated MiniPIX will measure the neutrons within the ISS structure in attempt to determine the exposure within the structure.

The materials that make-up the module will be those used in the outer walls on the ISS and will have the same thickness.
The main differences in structure will be the lack of standoffs and the lack of a multi-layer insulator (MLI) in our module.
The walls of the ISS have standoffs between layers of materials, which increases the total wall thickness to about one foot.
Additionally, the MLI used by NASA is not available commercially.
%Due to being in space, the gaps between the layers are a vacuum.
In the interest of saving space, we will not include the standoffs, and all materials will be layered on one another in the same order as they are used on the ISS.

\begin{figure}[H]
  \centering
  \includegraphics[width=0.8\linewidth]{Figures/MaterialCrossSection.png}
  \caption{Layers and thicknesses of the materials that will be used to construct the ISS module. From top to bottom: aluminum 6061-T6, six layers of Nextel AF62, six layers of Kevlar fabric, and aluminum 2219-T87. The atmosphere is contained by the \SI{4.8}{\milli\meter} layer of aluminum \cite{NASAPP}. All measurements are in \si{\milli\meter}.} 
  \label{fig:ISSLayers}
\end{figure}

%To construct the module, a rectangular box of aluminum will be constructed with one missing face, constituting the inner aluminum box that will contain the pressurized environment. The 

\subsubsection{Scintillated MiniPIX}
The MiniPIX's sensor can only interact with charged particles. In order to measure neutrons with the MiniPIX, a scintillator is rerquired. The scintillator will interact with the neutrons and produce a specific signature of particles which can be measured by the MiniPIX. Uher et al. \cite{Uher} used a similar method, but with a lithium-based scintillator. We will use a boron-loaded plastic scintillator \cite{BoronScintillator} to cover exactly half of the MiniPIX sensor. The reation between the Boron-10 nucleus and a thermal neutron is \[\ce{^10B + n -> ^7Li + ^4He + \gamma (\SI{480}{\kilo\eV})}\] according to Pawelczak et al \cite{Pawelczak}. This scintillator was chosen due to its large reaction cross-section for thermal neutron capture and its emission of light charged particles. This larger rwaction cross-section increases the probability that the neutron will produce particles that can be measured by the MiniPIX.

By covering half of the sensor, we can compare the uncovered half with the data recorded by the MiniPIX outside of the ISS module.

\subsubsection{Organic Photovoltaic Cells}    
We will be using four different types of solar cells with a total of 12 arrays on board our payload. There will be one type of cell on each of three sides of the payload, leaving one face free for any unique cells we may develop during our research. Half of our panels will be prefabricated and purcahsed from a supplier, and will be regarded as a control against which we will evaluate our in house fabricated cells. For the in house fabrication, we will be focusing on polymer:fullerne BHJ and perovskite active layer cells. We intend to investigate a variety of polymer:fullerene blends and develop a procedue for fabrication which yields the greatest PCE and MPP in the stratospheric environment, and to investigate the possibility of transpatent conductive oxide(TCO) layers besides the traditional indium tin oxide (ITO) glass and polyethylene terephthalate (PET) plastic. 

Solar cells are characterized by their current vs voltage (IV) curve, which tells you the maximum power point (MPP), Vmax and Imax, open-circuit voltage (Voc) and short circuit current (Isc), leading to the fill factor (FF) and efficiency (PCE). This curve will be  generated using a variable resistor in series with two multimeters, one acting as voltmeter and the other an ammeter, and the solar cell in question, which will follow Ohm's law (1) V=IR. From the plot, Voc is seen when I=0, Isc is when V=0, and the MPP is where Vmax and Imax meet. The MPP can then tell us the efficiecy, which is defined as (2) MPP/Pin. The fill factor gives information on the "squarness" of the IV curve and is defined as (3) FF = MPP/VocIsc, and it is optimal to maximize the fill factor.

    Before flight, every cell will be characterized with IV curves using AMG 1.5 simulated sun light to generate a characteristic profile for the each new cell, and microscopy will be preformed to understand the topology and interfacial details of the new cells. Currently our plan is to have one of each cell type remaining on ground as a control, and we will have each panel onboard the payload wired to an arduino unit where IV curves will be recorded during flight, with a characteristic profile generated every 30 seconds. This data will be stored on the arduino with backup stored to a raspberry pi. On  board our payload we will be measuring the tempreature and pressure of the environment to deteremaine any effects on the preformance due to temperature and pressure changes. Finally after recovery, the IV data will be extracted from the arduino/raspberry pi and plotted, and each cell will be re characterized and once again viewed under the microscope to look for any physical defects. 
    
    With the HASP mission, we are going to compare fabricated organic polymer and perovskite solar cells against purchased prefabricated organic solar cells. Each of these cells will be analyzed before, during, and after exposure to the stratosphere. In analyzing our cells, we will measure the open circuit voltage, short circuit current, and maiximum power point, then fill factor will be determined along with efficieny.
    In conjunction with our MiniPIX detector, we will be analyzing the effect that cosmic ray bombardment has on our solar cells. By analyzing the times at which cosmic rays are recorded, we can look at our IV data and determine if there was any effect on the cells preformance. Additionally, we will watch for problems which arise due to UV-A, B, and C rays with photodiodes. These photodiodes will serve the additional purpose of monitoring the solar power spectrum that our cells are exposed to.
    



    \subsection{SOCRATES Design}
\label{sec:ElectronicsDesign}

\subsubsection{Overview}

For this mission, we will use SOCRATES (System fOr Computing, Radiation, Astrobiology, Temperature, Environment and Solar), our flight computer, to manage flight operations and send and receive commands from the ground. SOCRATES is composed of two components: a Raspberry Pi 3 (RP3) and an Arduino Mega (Arduino) which interface via USB. SOCRATES' primary purposes during the flight will be to monitor environmental conditions inside the payload structure and inside the ISS module and to control the astrobiology system. It will also monitor temperature of the various subsystems including the solar cells, astrobiology collection system, MiniPIX, and air pressure throughout the flight. The temperature of each the 12 solar cells will be individually monitored with a thermistor. The voltage produced by each cell will also be monitored and recorded by SOCRATES via the Arduino. SOCRATES will use an accelerometer to monitor the activated state of the astrobiology system. All recorded data will be continuously written to an SD card mounted on a shield on top of the Arduino and to a removable USB flash drive connected to the RP3. SOCRATES will be programmed to deploy the astrobiology system at the appropriate altitude as sensed by the atmospheric pressure sensor. SOCRATES will also accept discrete commands from the HASP systems to turn the astrobiology collection system on and off.

In order to maintain data integrity we will save the data recorded by SOCRATES to both the SD card of the Arduino and to the USB drive connected to the Raspberry Pi. Data will be written to a discrete data file for a given period of time and then the file name will be incremented and the following data will be written to a new file. In case the data collection process is corrupted for a given period of time, this period of corruption will not interfere with the collection of all previous data. 

\subsubsection{The Sensors}

Our payload will utilize 14 thermistors to measure temperature at various points in our payload. The decision to use thermistors was based primarily on the performance of the analog temperature sensors during our 2017 flight, during which several of those sensors had slight malfunctions, and the success of thermistors in our 2018 flight. Thermistors are able to accurately measure temperature in the range \SIrange{-55}{125}{\celsius} and should therefore be adequate for the conditions in the stratosphere. Pressure will be recorded from two digital pressure sensors, one low pressure sensor in the main payload structure which will record pressure in the \SIrange{0}{3.44}{\kilo\pascal} range, and the second pressure sensor will be located inside the ISS module and record pressure in the \SIrange{20}{110}{\kilo\pascal} range. All sensor data will be UTC timestamped via the onboard real time clock and recorded to the SD card on the Arduino and USB drive attached to the RP3. To confirm the activated status of the astrobiology collection system, an accelerometer will be used to detect whether the collection arm is spinning or parked as is appropriate for the altitude of the payload. Each solar cell will also be monitored for temperature to aid in the calculation of efficiency of each cell in addition to being monitored for voltage directly through the analog ports of the arduino. Current of each solar cell will also be measured using a shunt resistor and IV curve and fill factor will be calculated after the payload has landed from the values that were collected for voltage and current of each cell. In order to accomodate all analog pins required to measure temperature, current, voltage and pressure, several analog multiplexes will be wired together in series for the arduino. One MiniPIX will be located inside the ISS module and an additional MiniPIX will be located inside the main structure of the payload.

\begin{table}[h!]
\centering
\caption{Sensors that compose SOCRATES}
\label{tab:Sensors}
\bigskip
\begin{tabular}{cccc}
  \hline
  \hline
  \multicolumn{1}{c}{\bfseries Sensor} & {\bfseries Quantity} & {\bfseries Platform} & {\bfseries Purpose} \\
  \hline
  Thermistor & 4 & Arduino & Record temperature measurements  \\
  Pressure (\SIrange{0}{3.44}{\kilo\pascal}) & 1 & Arduino & Record lower pressure measurements \\
  Pressure (\SIrange{20}{110}{\kilo\pascal}) & 1 & Arduino & Record higher pressure measurements \\
  Photodiodes & 3 & Arduino & Record intensity of visible light \\
  Accelerometer & 1 & RP3 & Report the current state of astrobiology system \\
  Real Time Clock & 2 & Arduino/RP3 & Record timestamps in UTC \\
  MiniPIX & 2 & RP3 & Cosmic ray detector \\
  \hline
  \hline
\end{tabular}
\end{table}


\subsubsection{Space Constraints}
Since half of the space inside the payload will be used by the astrobiology systems, we need to design our electronics to be relatively compact yet accessible for necessary modifications or repairs after testing. We will use one RP3 to both interface with two MiniPIX and store sensor and voltage data from the Arduino.  Also, in order to reduce the space required for the interface between the Arduino and all of the payload's sensors, we will use two layers of proto-shields to more effectively utilize vertical space. The RTC pressure sensors will be mounted directly on the first shield while the thermistors will be mounted on the top most shield. 

\subsubsection{Powering It Up}
 In order to stay within the power constraints, a robust power supply will need to  handle all the components of the payload.  The power supply we will be using is the PCM-DC-AT500 by WinSystems INC.  It offers one \SI{+5}{\volt} needed to power the payload's electronics.  This power supply could effectively take \SI{+30}{\volt} and step it up to the \SI{+12}{\volt} and \SI{+6}{\volt} outputs as needed. The power supply will also be able to step down to the \SI{+3.6 }{\volt} output as needed for sensors. One of the \SI{+12}{\volt} outputs goes to the Arduino since it can step down to the appropriate voltages internally while the other goes to a linear actuator for the astrobiology collection arm. Thermistors and pressure sensors will also receive power through the Arduino. A remaining \SI{+5 }{\volt} output is converted to a USB power cable for the RP3. 
 %More research needed regarding plans to ground components
 %The power supply also has four ground outputs that will be used by each respective component. 


\section{HASP Interface}
\label{sec:HaspInterface}

\subsection{Interfacing with HASP: Serial Uplink}
The SORA 3 payload will not utilize any serial commands. Everything will be configured such that the payload is autonomous.

\subsection{Interfacing with HASP: Serial Downlink}
For the duration of the flight, the serial downlink will be used to downlink the temperature and other various statistics that will be computed directly on the RP\num{3} as data is collected. It will also be used to downlink messages regarding the status of the payload, the command uplink status and error messages. The data packets will be human readable so they can be analyzed as they are received from HASP. The packets will be delimited by keywords \verb|begin_packet| and \verb|end_packet|, and we will use a simple one character header to differentiate between data packets and message packets. The format of the packets is potentially subject to change but the current design is outlined in Listing~\ref{Downlinks}.

Listing~\ref{Downlinks} is a sample of our proposed downlink data packet, values are included to the demonstrate expected number of characters but should not be considered expected or actual values. Included first is a format sample to enhance readability in this application which will not be included in the actual downlink data stream.

\lstset{basicstyle=\small, numbers=left, xleftmargin=2em, frame=tb, label = Downlinks, framexleftmargin=1.5em}
\begin{lstlisting}[caption = Sample of proposed downlink data packets ID: 15667 - 15669]
<Format Sample>
begin_packet
|ID Number|Raspberry Pi Temp|MiniPix 0 Temp|MiniPix 1 Temp|ISS Temp|ISS Pressure
|Accelerometer val x|Acc val y|Acc val z
|Solar Cell Index 00:Temp,FF,Efficiency|Solar Cell Index 01:Temp,FF,Efficiency
...
|Solar Cell Index 10:Temp,FF,Efficiency|Solar Cell Index 11:Temp,FF,Efficiency
|Timestamp|Date|
end_packet
<End Format Sample>
...
begin_packet
|15667|55.2|40.5|44.5|25.0|101.35
|1.00|-1.00|0.00
|C00:30.50,0.50,0.15|C01:30.50,0.50,0.15|C02:30.50,0.50,0.15|C03:30.50,0.50,0.15
|C04:30.50,0.50,0.15|C05:30.50,0.50,0.15|C06:30.50,0.50,0.15|C07:30.50,0.50,0.15
|C08:30.50,0.50,0.15|C09:30.50,0.50,0.15|C10:30.50,0.50,0.15|C11:30.50,0.50,0.15
|9:29:39|9/4/18|
end_packet
begin_packet
|15668|55.2|40.5|44.5|25.0|101.35
|1.00|-1.00|0.00
|C00:30.50,0.50,0.15|C01:30.50,0.50,0.15|C02:30.50,0.50,0.15|C03:30.50,0.50,0.15
|C04:30.50,0.50,0.15|C05:30.50,0.50,0.15|C06:30.50,0.50,0.15|C07:30.50,0.50,0.15
|C08:30.50,0.50,0.15|C09:30.50,0.50,0.15|C10:30.50,0.50,0.15|C11:30.50,0.50,0.15
|9:29:42|9/4/18|
end_packet
begin_packet
|15669|55.2|40.5|44.5|25.0|101.35
|1.00|-1.00|0.00
|C00:30.50,0.50,0.15|C01:30.50,0.50,0.15|C02:30.50,0.50,0.15|C03:30.50,0.50,0.15
|C04:30.50,0.50,0.15|C05:30.50,0.50,0.15|C06:30.50,0.50,0.15|C07:30.50,0.50,0.15
|C08:30.50,0.50,0.15|C09:30.50,0.50,0.15|C10:30.50,0.50,0.15|C11:30.50,0.50,0.15
|9:29:45|9/4/18|
end_packet
...
\end{lstlisting}
\medskip

The first integer represents the ID of the packet. Following the packet ID are temperature readings of the Raspberry Pi and both MiniPix, temperature and pressure within the ISS Module, astrobiology collection arm accelerometer $x, y, z$ values. Then follows the set of values for each of the twelve solar cells, which include the index, temperature, fill factor and efficiency of the cell. Included before the closing delimiter is the timestamp in $HH$:$MM$:$SS$ $MM$/$DD$/$YY$ at which the packet was written.  
During our previous flights, these data packets were crucial status updates to the state of our payload. We will once again use them for the same purpose to keep a close eye on our payload.

This year the data packets will contain information about the readings from temperatures on crucial devices such as the RP3 and MiniPIX sensors, as well as temperature and pressure inside the ISS module, and accelerometer $x, y, z$ values of the sample collection arm. Additionally packets will contain values regarding the temperature, fill factor, and efficiency of each of the solar cells.  


\subsection{Interfacing with HASP: EDAC and DB9 Connections}

The power supply unit (PSU) will use a +30V and ground line from the EDAC connection. The linear actuator of the collection system will use +12V and the collection arm servo motor will take a +6V connection. Other sensors will be powered through the Arduino taking +5V through the previously described USB connection.  

We will use discrete commands in this mission. Two of the commands will be used to deploy and retract the astrobiology system. This is only in case of emergency if we notice that the system has not already deployed at the planned altitude as designed. The other two discrete channels will be used to turn on and off SOCRATES in case that we notice an issue.

\begin{table}[!ht]
  \centering
  \caption{Table of all discrete commands to be used during flight} 
  \label{tab:Dis-Commands}
  \bigskip
  \begin{tabular}{cccc}
    \hline
    \hline
    \multicolumn{1}{c}{\bfseries Command} & \multicolumn{1}{c}{\bfseries Purpose} &  \multicolumn{1}{c}{\bfseries EDAC Pin} & \multicolumn{1}{c}{\bfseries Description} \\
    \hline
    Discrete 1 & Astro. System ON & f & Extends collection arm and activates rotation \\
    Discrete 2 & Astro. System OFF & n & Stops collection arm rotation and retracts arm \\
    Discrete 3 & SOCRATES ON & h & Powers up SOCRATES \\
    Discrete 4 & SOCRATES OFF & p & Shuts down SOCRATES \\
    \hline
    \hline
  \end{tabular}
  \medskip
\end{table}

\section{Thermal Control Plan}
\label{sec:TCP}


Based on our previous missions~\cite{SORA1}\cite{SORA2}, we have designed a thermal control plan centered around the high-performance devices. The devices that are at highest risk of thermal failure are the RP3 and the two MiniPIX devices. The data from the previous SORA missions strongly suggest that all devices will remain above their lower temperature limit. If a device were to experience thermal failure, it would be by overheating.

The temperature of all high-performance devices will be downlinked and monitored by the mission control team. In the case of a device overheating, said device will be shut down until it is safe to resume operations. The RP3 will be considered overheating if the temperature exceeds \SI{85}{\celsius}. A MiniPIX will be considered overheating if the device temperature exceeds \SI{80}{\celsius}.

In attempt to prevent overheating, each device will be configured with an individual metal heat sink. The RP3's processor will be equipped with a heat sink designed specifically for that device. Using the same technique used on the previous mission, each of the MiniPIX devices will be fixed to a plate of aluminum. For the MiniPIX contained in the pressurized environment, an aluminum heat pipe will be used to distribute heat from the device to the outer aluminum shell. 

\section{Power and Weight Budget}
\label{sec:PWBudget}

%In order to stay within the power constraints, a robust power supply will need to  handle all the components of the payload.  The power supply we will be using is the same PPM-DC-ATX-P by WinSystems INC that we used on our first flight~\cite{SORA}.  During our last flight it operated flawlessly and powered our payload throughout the whole mission.  It offers the desired number of \SI{+5}{\volt} and \SI{+12}{\volt} outputs needed to power the payload's electronics.  This power supply effectively takes \SI{+30}{\volt} and steps it down to two \SI{+12}{\volt} and two \SI{+5 }{\volt} outputs.  One of the \SI{+12}{\volt} outputs goes to the Arduino since it can step down to the appropriate voltages internally while the other goes to a PWM motor for the solenoid.  One of the \SI{+5 }{\volt} outputs powers two analog sensors that will be sent to HASP through the EDAC connection (more on that in the next sections).  The Radiation subsystem will be powered by batteries for the duration of the flight (see Appendix A).  The power supply also has four ground outputs that will be used by each respective component. 
\begin{table}[H]
  \centering
  \caption{Power and weight budget for SORA 3} 
  \label{tab:budget}
  \bigskip
  \begin{tabular}{cccccc}
    \hline
    \hline
    \multicolumn{1}{c}{\bfseries Component} & \multicolumn{1}{c}{\bfseries Voltage (VDC)} &  \multicolumn{1}{c}{\bfseries Current (mA)} & \multicolumn{1}{c}{\bfseries Duty Cycle (\%)} & \multicolumn{1}{c}{\bfseries Power (mW)} & \multicolumn{1}{c}{\bfseries Mass (g)} \\
    \hline
    30 to 5 V DC/DC Converter & 30 & 1500 & 100 & 45000 & 10 \\
    RP3 + (2)MiniPIX & 5 & 1210 & 100 & 6050 & 92.4 \\    
    Linear Actuator & 12 & 650 & 20 & 7800 & 84 \\
    Servo Motor & 6 & 550 & 100 & 3300 & 39 \\
    Accelerometer & 5 & 3.9 & 100 & 19.5 & 10 \\
    (14)Thermistors & 5 & 3.5 & 100 & 17.5 & 80 \\
    Altitude/Pressure Sensor & 3.6 & 1.4 & 100 & 5.04 & 10 \\
    Low Pressure Pressure Sensor & 5 & 7 & 100 & 35 & 10 \\   
    
    
    %Temperature Sensor 1 & 5 & 0.09 & 100 & 0.45 & 0.02 \\    
    %Polulu Driver & 30 & 1000 & CAL & CAL & 1 \\ \hline  
    %Pressure \& Altitude Sensor 1 & 3.3 & 1.4 & 100 & 4.62 & 0.005 \\ \hline
    %Real Time Clock (RTC) & 3.3 & 0 & 100 & 0.00198 & 0.005 \\ \hline
    %GPS & 3.3 & 41 & 100 & 135.3 & 0.005 \\ \hline
    %Real Time Clock (RTC) & 3.3 & 0 & 100 & 0.00198 & 0.005 \\ \hline
    %Humidity Sensor & 3.3 & 0.5 & 100 & 1.65 & 0.0025 \\ \hline
    Structure w/ bolts & N/A & N/A & N/A & N/A & 5000 \\
    \hline
    \textbf{Total} & \textbf{N/A} & \textbf{2427.6 peak} & \textbf{N/A} & \textbf{17234} & \textbf{5215} \\
    \hline
    \hline
  \end{tabular}
  \medskip
\end{table}

%Procedure: decontamination (pre and post flight)
\section{Procedures}
\label{sec:Procedures}

\subsection{Decontamination}
\label{subsec:Decontamination}

\subsubsection{Objectives}
Sanitization procedures are critical. They need to be checked and verified to ensure that our samples will not become contaminated. If the samples were to become contaminated it would make any possible bacterial collection data inconsequential.

\subsubsection{Sterilization Preflight}
The payload will be built within the confines of a class 100 clean hood that is located inside of a class 10,000 clean room. Any tools that are used to construct the sampling box will be heat sterilized at \SI{120}{\celsius} for \SI{20}{\minute}. This will be followed by exposing each side of the container to germicidal UV-C (\SI{254}{\nano\meter}) light for \SI{20}{\minute} and then soaked overnight in 91\% isopropyl alcohol to denature proteins in any possible sources of contaminating bacteria. This sterilization method destroys close to 100\% of all organisms and their endospores. To sterilize parts that would otherwise be damaged by the autoclave method they will be cleaned by hand with 91\% isopropyl alcohol to kill microorganisms by denaturing proteins and dissolving the lipid membrane. Following this, the materials will be rinsed with a 95\% ethanol (v/v) solution as an extra precautionary step to ensure complete decontamination. After all parts have dried, the sampling container will be constructed and placed in a gas-porous sterilization pouch and exposed to ethylene oxide (EO) at a concentration of 0.45-0.65 \SIrange{0.45}{0.65}{\milli\gram\per\meter\cubed} at \SI{55}{\celsius} and 30-50 \% RH for \num{4} hours to annihilate any spores and to provide another form of anti-bacterial treatment. The SMITH payload for HASP 2011 was processed in a similar fashion. Once the final HASP integration is ready for sampling and control containers are produced, the chambers will be sterilized and sealed. After the containers are integrated into the rest of the payload, the entire device will be placed in an autoclave bag for transportation.

\subsubsection{Sterilization Post flight}
Before payload descent, we will shut off all of our systesms. By powering down all the systems, the solenoids will seal the sampling container. Each team member involved in the recovery process will wear new latex gloves; cleaned with 91 \% isopropyl alcohol. The payload will remain sealed until decontamination procedures are complete and the sampling containers are ready for processing. The payload will be disassembled under class 100 conditions and all tools used during this procedure will be either heat or 91 \% isopropyl alcohol sterilized. Once in the clean room, the same procedures that were performed preflight will be performed post flight. The sampling box will then be packaged in a heat sterilized plastic outer container and transported back to the University of Houston for analysis.


\subsection{Testing and Integration}

\subsubsection{Vacuum Chamber Testing}
Each subsystem will be tested in-house with a vacuum chamber.
This testing phase will be used to ensure that each component of each subsystem performs as it should while in near-vacuum conditions, so if any problem arises it can be understood and fixed.
The ISS module structure will be tested individually with a pressure sensor to ensure that the module remains pressurized for a full \num{24} hour period which will ensure that the welded joints and the hermetic seal are sufficient.
The high-performance devices (i.e. the MiniPIXs and the RP3) will be thermally tested by performing and collecting data for a full \num{24} hour period.
Next, the full electronics system will be tested to ensure the astrobiology components correctly respond to the changing pressure.
To conclude vacuum testing, all subsytems will be fully integrated into one assembly which will then be tested in near-vacuum conditions for a \num{24} hour period.
The complete system will be considered optimal when both MiniPIX devices are ensured not to overheat, the RP3 is shown not to overheat, the astrobiology system deploys at the proper air pressure, and the astrobiology system seals at the proper air pressure all while remaining below the amperage limit of \SI{2.5}{\ampere}.

\subsubsection{HASP Integration}
During HASP integration, the payload will be tested for proper connection to the HASP systems and be thermally testing at least once more.
The integration team will verify that the EDAC connection distributes power as necessary as well as sends the correct discrete commands at will.
The astrobiology system will be tested without the membrane filters so as to conserve resources.
Any necessary tweaks or changes will be made to properly integrate our payload with the HASP systems.
The system will be considered integrated once the payload appropriately receives power and commands and completes thermal testing.

\subsubsection{Post-Integration Operations}
Once integration is complete, the subsystems will undergo one final check, and the astrobiology system will be sterilized and fitted with the membrane filters. If funds can support it, we will send a small team to New Mexico to oversee the flight operations.

\subsection{Flight Operations}
Each system is autonomous, however, a mission control team will monitor the downlinked data to determine the system status. If the astrobiology system fails to automatically respond or the electronics systems needs a reboot, the corresponding discrete command will be sent.

\subsection{Post-Flight Operations}
If a team is onsite in New Mexico, the data storage systems will be collected and safely stored. Otherwise, the payload will be shipped to our facilities.
                
%In-Flight Failure
\input{Sections/Failure.tex}

%Project Management includes the following: team structure, timeline, funding
\section{Project Management}
\label{sec:Management}

\subsection{Team Structure}
\label{sec:Team}

\emph{Faculty Mentor:}\par
\hspace{1cm} \textbf{Andrew Renshaw}\par
\hspace{1cm} Physics Department\par
\hspace{1cm} \url{arenshaw@central.uh.edu}\par
\vspace{.1in}

\emph{Project Leader:}\par
\hspace{1cm} \textbf{Reed Masek}\par
\hspace{1cm} Physics B.S., Spring 2020\par
\hspace{1cm} \url{rbmasek@uh.edu}\par
\vspace{.1in}

\emph{Team Coordinators:}\par
\hspace{1cm} \textbf{Jimish Patel}\par
\hspace{1cm} Physics B.S., Spring 2020\par
\hspace{1cm} \url{jimishpatel75@gmail.com}\par
\vspace{.05in}
\hspace{1cm} \textbf{Michael Butowicz}\par
\hspace{1cm}  Computer Science B.S., Fall 2020\par
\hspace{1cm} \url{mbutowicz@gmail.com}\par
\vspace{.05in}
\hspace{1cm} \textbf{Taylor Hill}\par
\hspace{1cm} Physics B.S., Spring 2019\par
\hspace{1cm} \url{tdhill92@gmail.com}\par

\begin{figure}[!h]
  \begin{center}
    \includegraphics[width=1\textwidth]{./Figures/TeamRoleTree.pdf}
    \caption{Team role tree for the current SORA 2.0 UH team.}
    \label{fig:Roles} 
  \end{center}
\end{figure}

\subsection{Roles and Responsibilities}
\label{sec:Roles}
\begin{itemize}
\item PI – Dr. Andrew Renshaw
	\begin{itemize}
	\item Attend weekly team meetings and provide general research team guidance
	\item Review project design and final products for submission to HASP
	\item Attend monthly teleconferences.
	\item Equipment procurement
	\end{itemize}
\item Project Leader - Reed Masek
	\begin{itemize}
	\item Interface with HASP Flight Control Team and act as team main point of contact
	\item Compile monthly reports and submit to HASP
	\item Attend monthly teleconference with HASP
	\item Coordinate with PI on administration tasks and internal group business
        \item Periodically check-in with team coordinators regarding the team's progress
        \item Coordinate meetings and assign tasks with deadlines
	\item Approve designs, tests, ideas, and any work related to HASP and payload
	\item Final decisions on staffing (staffing decisions will be a group decision overall)
	\end{itemize}
\item Electronics and Communications Coordinator - Michael Butowicz
	\begin{itemize}
	\item Do necessary reserach for finalizing work
	\item Coordinate information, tasks, and deadlines with subgroup
	\item Approve work done by subsystem team
	\item Write bimonthly updates along with detailed reports from subsytem meetings
	\item Make detailed presentations, if necessary, for weekly team meetings
	\item Report to project leader and PI with any project changes, issues encountered, and any external communications.
	\item CC PI and Project Leader in all emails for external communications
	\item Perform other such duties as the Project Leader or PI may specify
	\end{itemize}
\item Astrobiology Coordinator - Jimish Patel
	\begin{itemize}
	\item Do necessary reserach for finalizing work
	\item Coordinate information, tasks, and deadlines with subgroup
	\item Approve work done by subsystem team
	\item Write bimonthly updates along with detailed reports from subsytem meetings
	\item Make detailed presentations, if necessary, for weekly team meetings
	\item Report to project leader and PI with any project changes, issues encountered, and any external communications.
	\item CC PI and Project Leader in all emails for external communications
	\item Perform other such duties as the Project Leader or PI may specify
	\end{itemize}
\item Radiation Coordinator - Taylor Hill
	\begin{itemize}
	\item Do necessary reserach for finalizing work
	\item Coordinate information, tasks, and deadlines with subgroup
	\item Approve work done by subsystem team
	\item Write bimonthly updates along with detailed reports from subsytem meetings
	\item Make detailed presentations, if necessary, for weekly team meetings
	\item Report to project leader and PI with any project changes, issues encountered, and any external communications.
	\item CC PI and Project Leader in all emails for external communications
	\item Perform other such duties as the Project Leader or PI may specify
	\end{itemize}
\item Team Member
	\begin{itemize}
	\item Do necessary reserach for finalizing work
	\item Coordinate with team and subsystem coordinator
	\item Make detailed presentations, if necessary, for weekly team meetings
	\item Report to project leader and PI with any project changes, issues encountered, and any external communications.
	\item CC PI and Project Leader in all emails for external communications
	\item Perform other such duties as the Project Leader or PI may specify
	\end{itemize}
\end{itemize}


\subsection{Timeline}
\label{sec:Timeline}
\begin{table}[H]
\centering
\caption{Timeline for the 2019 SORA 3 Mission}
\label{timeline}
\resizebox{\textwidth}{!}{%
\begin{tabular}{|c|l|}
\hline
\textbf{Month of 2019} & \multicolumn{1}{c|}{\textbf{Description of Work}} \\ \hline
\textbf{January} & \begin{tabular}[c]{@{}l@{}} * Secure funding \\ * Create and finish budget for mission \\ * Make inventory of hardware \\ * Procure hardware/software \\ * Start designs of SORA 3 \\ * Update SOCRATES \\ * Upgrade vacuum chamber \\ * Recruit new members \end{tabular} \\ \hline
\textbf{February} & \begin{tabular}[c]{@{}l@{}}
* Continue with work from January in terms of design development and funding procurement.  \\ * Have finished list of inventory\\ * Continue recruitment if necessary and finalize by end of month.\\\end{tabular} \\ \hline
\textbf{March} & \begin{tabular}[c]{@{}l@{}}Obtain funding by the end of this month.  Finish all tasks from the previous two months and transition into building phase.\\ * Have all hardware/software orders in by the end of the month\\ * Begin PSIP, have draft by end of the month\end{tabular} \\ \hline
\textbf{April} & \begin{tabular}[c]{@{}l@{}}* Order remaining items if needed\\ * Submit a NASA On-site Security Clearance Document by April 15th.\\ * Finish PSIP by April 26th\\ * SOCRATES and MiniPIX integration and testing\\ * Prepare for astrobiology work\end{tabular} \\ \hline
\textbf{May} & \begin{tabular}[c]{@{}l@{}}* PSIP and FLOP development\\ * Finalize integration of SOCRATES and hardware\\ * Continue working on astrobiology upgrades\end{tabular} \\ \hline
\textbf{June} & \begin{tabular}[c]{@{}l@{}}Final PSIP due June 28th\\ *Finalize astrobiology upgrades and ready for integration\\ * Testing in lab\end{tabular} \\ \hline
\textbf{July} & \begin{tabular}[c]{@{}l@{}}Final FLOP due July 19th\\ * Make changes from testing and continue tests\end{tabular} \\ \hline
\textbf{August} & \begin{tabular}[c]{@{}l@{}}Payload Integration during July 15th - July 19th\\ *Have all payload work done and ready for flight\end{tabular} \\ \hline
\textbf{September} & \begin{tabular}[c]{@{}l@{}}* Launch SORA 3 on September 2nd\\ * Recovery TBD. \end{tabular} \\ \hline
\textbf{October} & Debrief and analyze all data from flight \\ \hline
\textbf{November} & Have final report by end of November \\ \hline
\textbf{December} & Final Report due on the 6th \\ \hline
\end{tabular}%
}
\end{table}


\subsection{Funding}
\label{sec:Funding}
Funding sources will be local contributions - applying for funding through the Physics Department, the College of Natural Science and Mathematics, The University of Houston Division of Research, and local organizations and companies willing to support this endeavor.  Additionally, the University of Houston Department of Biology has promised long-term support for development of the astrobiology system and analysis of the astrobiology data. 


\newpage
\section{Appendix A}
\label{sec:Appendix A}

\subsection{Payload Dimensions}

	
\newpage

\begin{thebibliography}{9}
\bibitem{SORA1}
S.A. Garcia Morelos, F. Brooks, S. Oliver, A. Walker, K.D. Portillo, R.B. Masek, D. Mroczek, D. Pena, J. Juarez, A. Cruz, D. Henandez, S. George, D. Pattison, A.L. Renshaw. \textit{Scientific Report for the 2017 UH Team.} SORA 2017 Mission Webpage. \url{http://laspace.lsu.edu/hasp/groups/Payload.php?py=2017&pn=10}.

\bibitem{SORA2}
  S. A. Garcia Morelos, F. Brooks, S. Oliver, A. Walker, K. D. Portillo, R. B. Masek, J. Patel, S. George, I. Wilson, D. Pattison, P. Gunaratne, and A. L. Renshaw. \textit{SORA 2.0: Stratospheric Organism and Radiation Analyzer}

\bibitem{MiniPIX}
  MiniPIX - Miniaturized Portable USB Photon Counting Camera. (n.d.). Retrieved February 02, 2017, from \url{http://advacam.com/camera/minipix}.

\bibitem{Regener}
 Regener E. \& Pfotzer G., \textit{Vertical Intensity of Cosmic Rays by Threefold Coincidences in the Stratosphere.}, Nature 136, 718-719, (1935). 
  
\bibitem{LSU}
  Christner, B., Alleman, M., Bryan, N., Burke, S., Guzik, T.G., Granger, D., King, G. (2013) \textit{LSU HASP2013 PDF. Baton Rouge: Louisiana Space Consortium}.

\bibitem{Extremophiles}
  Extremophiles \href{http://www.nytimes.com/2013/02/07/science/living-bacteria-found-deep-under-antarctic-ice-scientists-say.html}{http://www.nytimes.com/2013/02/07/science/living-bacteria-found-deep-\\under-antarctic-ice-scientists-say.html}.

\bibitem{Frank}
  Schröder, F. G. (2017). Radio detection of cosmic-ray air showers and high-energy neutrinos. Progress in Particle and Nuclear Physics, 93, 1-68. doi:10.1016/j.ppnp.2016.12.002

\bibitem{GCRs}
  https://helios.gsfc.nasa.gov/gcr.html

\bibitem{NASA HRP}
  https://www.nasa.gov/hrp

\bibitem{NASAShielding}
  National Research Council. 1997. \textit{Protecting the Space Station from Meteoroids and Orbital Debris}. Washington, DC: The National Academies Press. \url{https://doi.org/10.17226/5532}.

\bibitem{NASAPP}
  Christiansen, E. L., Lear, D. M. (2012). Micrometeoroid and Orbital Debris Environment \& Hypervelocity Shields [PowerPoint slides]. Retrieved from \url{https://ntrs.nasa.gov/archive/nasa/casi.ntrs.nasa.gov/20120002584.pdf}.
  
\bibitem{SolarEnergyMaterials}
  Solar Energy Materials and Solar Cells 182 (2018) 121–127 DOI: 10.1016/j.solmat.2018.03.024.

\bibitem{Kaltenbrunner}
  Kaltenbrunner, M. et al. Ultrathin and lightweight organic solar cells with high flexibility. Nat. Commun. 3:770 DOI: 10.1038/ncomms1772 (2012).

\bibitem{Uher}
  Uher, J., Frojdh, C., Holy, T., Jakubek, J., Petersson, S., Pospisif, S., . . . Stekl, I. (2007). Silicon Detectors for Neutron Imaging. AIP Conference Proceedings. doi:10.1063/1.2825756

\bibitem{BoronScintillator}
  EJ-254 Boron Loaded Plastic Scintillator \url{https://eljentechnology.com/images/products/data_sheets/EJ-254.pdf}.

\bibitem{Pawelczak}
  Pawełczak, I., Glenn, A., Martinez, H., Carman, M., Zaitseva, N., \& Payne, S. (2014). Boron-loaded plastic scintillator with neutron-$\gamma$ pulse shape discrimination capability. Nuclear Instruments and Methods in Physics Research Section A: Accelerators, Spectrometers, Detectors and Associated Equipment, 751, 62-69. doi:10.1016/j.nima.2014.03.027
  
\bibitem{Canales}
 Canales D. C. and Ehteshami A., \textit{An attempt to sample atmospheric bacteria}, Houston, TX, 2015, January 11.

\bibitem{Bexus}
Urbar, J., Scheirich, J., Jakubek, J., 2011. Medipix/Timepix cosmic ray tracking on BEXUS stratospheric balloonflights. Nucl. Instrum. Methods A 633, S206-209.

\bibitem{Advacam}
  ADVACAM at \url{advacam.com}.

\bibitem{Medipix}
  Medipix collaboration at \url{https://medipix.web.cern.ch/}.
  
\bibitem{StuartThesis} 22
  George, S., \textit{Dosimetric Applications of Hybrid Pixel Detectors}, University of Wollongong, Australia, 2015.
  
  \bibitem{OPV operation}
  DOI: 10.1038/nmat3807 


%\bibitem{uv_irradiance}
%  Calculating the UV Index. (2016, October 14). Retrieved June 03, 2017, from \url{https://www.epa.gov/sunsafety/calculating-uv-index-0}.

%\bibitem{cleanbox}
% Clean box material \url{https://www.mcmaster.com/\#uhmw-polyethylene/=1aijn1p}.

%\bibitem{Valve}
 %Valve data sheet \url{http://www.generant.com/Literature/Series\%20VRV\%20Product\%20Literature.pdf}.

%\bibitem{mpdatasheet}
%  ADVACAM, \textit{MINIPIX Version 1.0 Datasheet}, Retrieved from \url{http://www.widepix.cz/files/datasheets/MiniPIX\%20v1.0\%20Datasheet.pdf}.

%\bibitem{mpjakubek}
%  Jan Jakubek, \textit{Precise energy calibration of pixel detector working in time-over-threshold mode} Institute of Experimental and Applied Physics, Czech Technical University in Prague, Czech Republic, 2011.
  
%\bibitem{magnetictool}
%  United States National Oceanic and Atmospheric Administration, \textit{Magnetic Field Calculators} [Data sets], Retrieved from \url{https://www.ngdc.noaa.gov/geomag-web/#igrfwmm}.

%\bibitem{gorman}
%	Gorman, J. (2013, February 06). \textit{Scientists Find Life in the Cold and Dark Under Antarctic Ice.} Retrieved September 15, 2016, from Scientists Find Life in the Cold and Dark Under Antarctic Ice.
 
%\bibitem{pumpsource}
%  \url{http://www.knfusa.com/?type=5600&amp;file=2079}.

%\bibitem{Horneck}
%  Horneck, G. 1993. The Biostack concept and its application in space and at accelerators: studies in Bacillus subtilis spores, p. 99-115. In C. E. Swenberg, G. Horneck, and E. G. Stassinopoulos (ed.), \textit{Biological effects and physics of solar and galactic cosmic radiation}[PDF], part A. Plenum Press, New York, NY. accessed 10/24/16  

%\bibitem{Horneck} 
%  Horneck, G. 2007. \textit{Space radiation biology}[PDF], p. 243-273. In E. Brinckmann (ed.), Biology in space and life on Earth. Wiley-VCH, Weinheim, Germany. Accessed 10/26/16

%\bibitem{Horneck}
%  Horneck, G., C. Baumstark-Khan, and G. Reitz. 2002. \textit{ Space microbiology: effects of ionizing radiation on microorganisms in space}[PDF], p. 2988-2996. In G. Bitton (ed.), The encyclopedia of environmental microbiology. John Wiley \& Sons, New York, NY. Accessed 10/30/16

%\bibitem{Horneck}
%  Horneck, G., C. Baumstark-Khan, and R. Facius. 2006. \textit{Radiation biology}[PDF], p. 292-335. In G. Cl?ment and K. Slenzka (ed.), Fundamentals of space biology. Kluwer Academic Publishers/Springer, Dordrecht, The Netherlands. accessed 11/4/16

%\bibitem{Kiefer}
%Kiefer, J., K. Schenk-Meuser, and M. Kost. 1996. \textit{Radiation biology}[PDF], p. 300-367. In D. Moore, P. Bie, and H. Oser (ed.), Biological and medical research in space. Springer, Berlin, Germany. accessed 11/9/16

%\bibitem{SamURD}
 % Alfonso Garcia Morelos, S. (2016, October 13).
 % \textit{A Novel Microbe Trap.}
 % Presentation at UH Undergraduate Research Day. \url{http://www.uh.edu/honors/undergraduate-research/}
  
%\bibitem{SamAPS}
%	Alfonso Garcia Morelos, S. (2017, October 20).
%	\textit{Stratospheric Organism and Radiation Analyzer}
%	Retrieved October 20, 2017, from \textit{Bulletin of the American Physical Society}. \url{https://meetings.aps.org/Meeting/TSF17/Session/E5.3}
	
%\bibitem{StevenURD}
%  Oliver, S. J. (2017, October 12). 
%  \textit{Stratospheric Organism and Radiation Analyzer}
%  Presentation at UH Undergraduate Research Day. Retrieved October 12, 2017, from \url{http://www.uh.edu/honors/undergraduate-research/events/urday2017/}

%\bibitem{StevenSchoolPres}
%  Oliver, S. J. (2017, November 4). 
%  \textit{STEM Life at UH}
%  Presentation at UH Gathering of the Eagles STEM Symposium. \url{https://www.uh.edu/news-events/stories/2016/November/110416EaglesSTEM.php}

%\bibitem{Fre}
%  Brooks, F. (2017, January 14).
%  \textit{Stratospheric Organism and Radiation Analyzer}
%  Presentation at Rice University, APS Conferences for Undergraduate Women in Physics (CUiP). \url{http://www.google.com/url?q=http%3A%2F%2Fwww.aps.org%2Fprograms%2Fwomen%2Fworkshops%2Fcuwip.cfm&sa=D&sntz=1&usg=AFQjCNE5pImV-SVrb87CvgAa9RSfeCrYXg}  
  
\end{thebibliography}



\end{document}
